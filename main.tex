% !TeX spellcheck = sk_SK
% LaTeX document class
\documentclass[nominted]{uniza}

%-------------------------------------------------------
%             Abbreviation and term database
%-------------------------------------------------------
\input{modules/abbterms.tex}

%-------------------------------------------------------
%            Súbory s bibliografickými informáciami
%-------------------------------------------------------
\addbibresource{bibliography.bib}

%-------------------------------------------------------
%                  Jazykové nastavenia
%-------------------------------------------------------
\usepackage[english,slovak]{babel}

%-------------------------------------------------------
%                Potrebné balíčky
%-------------------------------------------------------

\usepackage{etoc}

%-------------------------------------------------------
%                Informácie o dokumente
%-------------------------------------------------------


\title{Názov práce}
\subtitle{Podnázov práce (ak existuje)}
\author{Jozef Mrkva}
\keywords{Vložte minimálne 4 kľúčové slová výstižne charakterizujúce spracovanú tému.}
\keywordsSecLang{Insert the minimum of 4 keywords that accurately characterize your topic.}

\keywordsName{Kľúčové slová}
\keywordsNameSecLang{Keywords}

\faculty{Elektrotechnická fakulta}
\department{Katedra riadiacich a informačných systémov}

% školiace pracovisko, ak je iné než katedra:
%\supervisorinst{
%	SIEMENS\\
%	Kompetenčné centrum Žilina
%}

\facultyshort{EF}
\location{Žilina}

\doctype{Diplomová práca}
\docid{Evidenčné číslo práce}
\supervisor{Titul, vedúci práce}
%Meno konzultanta (ak existuje):
\consultant{Titul, konzultant práce}

\academicyear{2018/2019} % akademický rok
\submissionyear{2019} % rok odovzdania práce
% Študijný program:
\studyprogramme{Riadenie procesov}
% Študijný odbor:
\fieldofstudy{5.2.14 Automatizácia}

% Abstrakt v hlavnom jazyku
\abstract{Abstrakt}{
Abstrakt obsahuje informáciu o cieľoch práce, jej stručnom obsahu a v závere abstraktu sa charakterizuje splnenie cieľa, výsledky a význam celej práce. Abstrakt sa píše súvisle ako jeden odsek a jeho rozsah je spravidla 100 až 500 slov.
}

% Abstrakt v cudzom jazyku (anglickom, nemeckom, ...)
\abstractSecLang{Abstract}{
In this place insert the text of the abstract in English or another foreign language. Sem vložte text abstraktu v angličtine, prípadne v inom cudzom jazyku.
}

%Dátum odovzdania práce
\date{Dátum odovzdania práce}

\acknowledgements{
	Poďakovanie nie je povinné. Ak nemá byť zahrnuté, stačí túto časť zakomentovať.
}

%-------------------------------------------------------
%		 Vybrané metadáta zapíšeme aj do dokumentu.
%-------------------------------------------------------

\hypersetup{
	pdfauthor={\Author},%
    pdftitle={\Title},%
    pdfsubject={\Doctype},%
    pdfkeywords={\Keywords},%
    pdfproducer={LaTeX},%
%    pdfcreator={pdfLaTeX}
}

%-------------------------------------------------------
%						Includeonly
%-------------------------------------------------------

%\includeonly{
%kap_uvod
%}

%-------------------------------------------------------
%		Korektné zalamovanie spojok na konci riadku.
%-------------------------------------------------------
\usepackage{encxvlna}
% a temporary workaround for encxvlna breaking multiarg macros
\mubyte { {\endmubyte

%-------------------------------------------------------
%					Začiatok dokumentu
%-------------------------------------------------------

\begin{document}

%-------------------------------------------------------
%				 Obálka a titulná strana
%-------------------------------------------------------

\makecover
\maketitle

%-------------------------------------------------------
%						 Zadanie
%-------------------------------------------------------

\includepdf[fitpaper]{modules/zadanie.pdf}

%-------------------------------------------------------
%				Front Matter (TOC, LOF, ...)
%-------------------------------------------------------
\frontmatter

% suppress some commands in TOC and lists
\begingroup
\renewcommand{\ac}[1]{#1}
\renewcommand{\cite}[1]{}

% Poďakovanie
\makeacknowledgements

% Abstrakt, anotácia
\makeabstract

%  TOC
\etocdepthtag.toc{mtchapter}
\etocsettagdepth{mtchapter}{subsection}
\etocsettagdepth{mtappendix}{none}
\tableofcontents

% list of figures
\etocdepthtag.toc{mtfigure}
\etocsettagdepth{mtfigure}{subsection}
\etocsettagdepth{mtfigure}{none}
\iftotalfigures\listoffigures\fi

% list of tables
\etocdepthtag.toc{mttable}
\etocsettagdepth{mttable}{subsection}
\etocsettagdepth{mttable}{none}
\iftotaltables\listoftables\fi

% list of abbreviations
\acsetup{page-ref=comma,list-style=acronyms
}{
	\ifargoutempty{\printacronyms[include-classes=abbrev,heading=none]}{}{
		\unchapter{Zoznam skratiek}
		\argoutemptyFirstArg
	}
}

% dictionary
\acsetup{page-ref=none,list-style=dictstyle,extra-style=plain,extra-format={\;},only-used=false}{
\ifargoutempty{\printacronyms[include-classes=dict,heading=none]}{}{
	\unchapter{Slovník pojmov}
	\setlength{\columnseprule}{0.2pt}
	\setlength{\columnsep}{1.25cm}
	\begin{multicols}{2}
	\argoutemptyFirstArg
	\end{multicols}
}}

\endgroup % suppress some commands in TOC and lists

%-------------------------------------------------------
%						Document
%-------------------------------------------------------
\mainmatter

\include{kap_uvod}
\include{kap_sablona}
\include{kap_latex}
\include{kap_citacie}
\include{kap_typografia}

%-------------------------------------------------------
%                     Bibliography
%-------------------------------------------------------

\printbibliography[heading=unchapter,title={Zoznam použitej literatúry}]

%-------------------------------------------------------
%					Čestné vyhlásenie
%-------------------------------------------------------

\makeDeclaration

%-------------------------------------------------------
%						Appendix
%-------------------------------------------------------

\makeAppendixPage
\appendix

\etoctocstyle{1}{Zoznam príloh}
\etocdepthtag.toc{mtappendix}
\etocsettagdepth{mtchapter}{none}
\etocsettagdepth{mtappendix}{subsection}
\tableofcontents

\include{priloha1}

\end{document}
