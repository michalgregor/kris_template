% !TeX spellcheck = sk_SK
% LaTeX document class
\documentclass{kris}

%-----------------------------------------------------------
%             Abbreviation and term database
%-----------------------------------------------------------
\input{modules/abbterms.tex}

%-----------------------------------------------------------
%            Súbory s bibliografickými informáciami
%-----------------------------------------------------------
\addbibresource{bibliography.bib}

%-----------------------------------------------------------
%                  Jazykové nastavenia
%-----------------------------------------------------------
\usepackage[english,slovak]{babel}

%----------------------------------------------------------
%                Informácie o dokumente
%----------------------------------------------------------

\title{Názov práce}
\author{Titul, meno a priezvisko}
\keywords{Vložte minimálne 4 kľúčové slová výstižne charakterizujúce spracovanú tému.}
\keywordsSecLang{Insert the minimum of 4 keywords that accurately characterize your topic.}

\keywordsName{Kľúčové slová}
\keywordsNameSecLang{Keywords}

\doctype{Diplomová práca}
\docid{Evidenčné číslo práce}
\supervisor{Titul, vedúci práce}
\consultant{Titul, konzultant práce}
\reviewer{Titul, recenzent práce}
\academicyear{2017/2018}

% Abstrakt v hlavnom jazyku
\abstract{Abstrakt}{
Abstrakt obsahuje informáciu o cieľoch práce, jej stručnom obsahu a v závere abstraktu sa charakterizuje splnenie cieľa, výsledky a význam celej práce. Abstrakt sa píše súvisle ako jeden odsek a jeho rozsah je spravidla 100 až 500 slov.
}

% Abstrakt v cudzom jazyku (anglickom, nemeckom, ...)
\abstractSecLang{Abstract}{
In this place insert the text of the abstract in English or another foreign language. Sem vložte text abstraktu v angličtine, prípadne v inom cudzom jazyku.
}

% Anotácia v hlavnom jazyku
\annotation{Anotácia v slovenskom jazyku}{
Anotácia je krátka charakteristika, stručný popis informačného zdroja nasledujúci bibliografickú citáciu. Anotácia môže obsahovať sumár kľúčových bodov zdroja, opis druhu zdroja a jeho ohodnotenie. Na rozdiel od abstraktu, čo je krátky objektívny popis zdroja, anotácia je kritické alebo subjektívne hodnotenie diela zvyčajne zahrnuté v bibliografii alebo citácii.
}

% Anotácia v cudzom jazyku (anglickom, nemeckom, ...)
\annotationSecLang{Anotácia v anglickom jazyku}{
Anotácia je krátka charakteristika, stručný popis informačného zdroja nasledujúci bibliografickú citáciu. Anotácia môže obsahovať sumár kľúčových bodov zdroja, opis druhu zdroja a jeho ohodnotenie. Na rozdiel od abstraktu, čo je krátky objektívny popis zdroja, anotácia je kritické alebo subjektívne hodnotenie diela zvyčajne zahrnuté v bibliografii alebo citácii.
}

\date{Dátum odovzdania práce}

\acknowledgements{
	Poďakovanie nie je povinné. Ak nemá byť zahrnuté, stačí túto časť zakomentovať.
}

%----------------------------------------------------------
%		 Vybrané metadáta zapíšeme aj do dokumentu.
%----------------------------------------------------------

\hypersetup{
	pdfauthor={\Author},%
    pdftitle={\Title},%
    pdfsubject={\Doctype},%
    pdfkeywords={\Keywords},%
    pdfproducer={LaTeX},%
%    pdfcreator={pdfLaTeX}
}

%----------------------------------------------------------
%						Includeonly
%----------------------------------------------------------

%\includeonly{
%kap_uvod
%}

%----------------------------------------------------------
%		Korektné zalamovanie spojok na konci riadku.
%----------------------------------------------------------
\usepackage{encxvlna}

%----------------------------------------------------------
%					Začiatok dokumentu
%----------------------------------------------------------

\begin{document}

%----------------------------------------------------------
%				 Obálka a titulná strana
%----------------------------------------------------------

\makecover
\maketitle

%----------------------------------------------------------
%						 Zadanie
%----------------------------------------------------------

\includepdf[fitpaper]{modules/zadanie.pdf}

%----------------------------------------------------------
%				Front Matter (TOC, LOF, ...)
%----------------------------------------------------------
\frontmatter
% !TeX spellcheck = sk_SK

% suppress some commands in TOC and lists
\begingroup

\renewcommand{\ac}[1]{#1}
\renewcommand{\cite}[1]{}

%----------------------------------------------------------
%                        TOC
%----------------------------------------------------------
\tableofcontents

%----------------------------------------------------------
%                    List of Abbrs
%----------------------------------------------------------

\acsetup{page-ref=comma,list-style=acronyms
}{
	\ifargoutempty{\printacronyms[include-classes=abbrev,heading=none]}{}{
		\unchapter{Zoznam skratiek}
		\FirstArg
	}
}

%----------------------------------------------------------
%                    List of Terms
%----------------------------------------------------------

\newlist{dict}{description}{1}
\setlist[dict]{
	style=unboxed,
	labelindent=0pt,
	itemindent=0pt,
	listparindent=0pt,
  	leftmargin=0cm,
  	font=\addcolon
}
\DeclareAcroListStyle{dictstyle}{list}{ list = dict }

\newlist{subdict}{description}{1}
\setlist[subdict]{
	style=nextline,
	leftmargin=0cm,
	labelindent=0cm,
	itemindent=1cm,
	listparindent=0pt,
  	font=\normalfont\textit
}

\acsetup{page-ref=none,list-style=dictstyle,extra-style=plain,extra-format={\;},only-used=false}{
\ifargoutempty{\printacronyms[include-classes=dict,heading=none]}{}{
	\unchapter{Slovník pojmov}
	\setlength{\columnseprule}{0.2pt}
	\setlength{\columnsep}{1.25cm}
	\begin{multicols}{2}
	\FirstArg
	\end{multicols}
}}

%----------------------------------------------------------
%                    List of Figures
%----------------------------------------------------------
\iftotalfigures\listoffigures\fi

%----------------------------------------------------------
%                    List of Tables
%----------------------------------------------------------
\iftotaltables\listoftables\fi

\endgroup % suppress some commands in TOC and lists

%----------------------------------------------------------
%				 Abstrakt, anotácia
%----------------------------------------------------------

\makeabstract
\makeannotation

%----------------------------------------------------------
%					Poďakovanie.
%----------------------------------------------------------

\makeacknowledgements

%----------------------------------------------------------
%						Document
%----------------------------------------------------------
\mainmatter

\include{kap_uvod}
\include{kap_sablona}
\include{kap_latex}
\include{kap_citacie}
\include{kap_typografia}

%----------------------------------------------------------
%                     Bibliography
%----------------------------------------------------------

\printbibliography[heading=unchapter,title={Zoznam použitej literatúry}]

%----------------------------------------------------------
%					Čestné vyhlásenie
%----------------------------------------------------------

\makeDeclaration

%----------------------------------------------------------
%						Appendix
%----------------------------------------------------------

\makeAppendixPage
\appendix

\include{priloha1}

\end{document}
